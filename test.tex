\documentclass[11pt, draft]{article}

\usepackage[utf8]{inputenc}
\usepackage[margin=35truemm]{geometry} % 余白を35mmに設定

% --- 数学関連のパッケージ ---
\usepackage{amsmath}      % 様々な数式環境
\usepackage{amsthm}       % 定理環境
\usepackage{amsfonts}     % 数学用フォント
\usepackage{physics}      % 物理学の数式用マクロ (bra, ket, dvなど)
\usepackage{bm}           % 太字の数式 \bm{}

% --- 図や色、レイアウト関連のパッケージ ---
\usepackage[dvipdfmx]{graphicx, color} % 図の挿入や色の利用
\usepackage{tikz}         % 図形描画
\usetikzlibrary{intersections,calc,arrows.meta}
\usepackage{float}        % 図表の位置を調整 [H]など
\usepackage{siunitx}      % 単位をきれいに表示 \SI{100}{\kilo\gram}など
\usepackage{ascmac}       % itemboxなどの囲み枠

% --- その他 ---
\parindent = 0pt % 全体の段落開始時のインデントをなくす

% --- ハイパーリンク関連 (できるだけ最後に読み込む) ---
\usepackage[
    dvipdfmx,
    bookmarks=true,
    bookmarksnumbered=true,
    bookmarkstype=toc
]{hyperref}
\usepackage{pxjahyper} % hyperrefの日本語文字化け対策
\begin{document}

\title{幾何学}
\author{小武龍斗 22B20817}
\date{\today}
\maketitle

\section{問題1}
\noindent
問1. $V, W$ を (可換) 体 $K$ 上の有限次元ベクトル空間とし, $\varphi: V \longrightarrow W$ を線型写像とする. $\{e_1, \dots, e_n\}$ を $V$ の基底とし, $\{e^{*1}, \dots, e^{*n}\}$ をその双対基底とする. また, $\{f_1, \dots, f_m\}$ を $W$ の基底とし, $\{f^{*1}, \dots, f^{*m}\}$ をその双対基底とする.

\vspace{0.5em}

このとき, もし基底 $\{e_1, \dots, e_n\}, \{f_1, \dots, f_m\}$ に関する $\varphi$ の表現行列が $A$ であれば, 基底 $\{f^{*1}, \dots, f^{*m}\}, \{e^{*1}, \dots, e^{*n}\}$ に関する ${}^t\varphi$ の表現行列は ${}^tA$ になることを示せ. 但し ${}^t\varphi$ は次式で定義される線型写像 ${}^t\varphi: W^* \longrightarrow V^*$ である.
\[
  {}^t\varphi(g) = g \circ \varphi \quad \text{for } g \in W^*.
\]
\subsection{解答}
まず、表現行列の定義を考える。
\begin{align}
    \varphi(e_j) = \sum_{i=1}^{m} A_{ij} f_i
\end{align}
行列形式で書くと、
\begin{align}
    \begin{pmatrix}
        \varphi(e_1) & \varphi(e_2) & \cdots & \varphi(e_n)
    \end{pmatrix}
    =
    \begin{pmatrix}
        f_1 & f_2 & \cdots & f_m
    \end{pmatrix}
    A
\end{align}
次に、${}^t\varphi$ の表現行列を $B$ とする。
\begin{align}
    {}^t\varphi(f^{*i}) = \sum_{j=1}^{n} B_{ji} e^{*j}
\end{align}
行列形式で書くと、
\begin{align}
    \begin{pmatrix}
        {}^t\varphi(f^{*1}) & {}^t\varphi(f^{*2}) & \cdots & {}^t\varphi(f^{*m})
    \end{pmatrix}
    =
    \begin{pmatrix}
        e^{*1} & e^{*2} & \cdots & e^{*n}
    \end{pmatrix}
    B\label{4}
\end{align}
ここで、${}^t\varphi(f^{*i})$ を定義

に従って書き換える。
\begin{align}
    {}^t\varphi(f^{*i})(e_j) &= f^{*i}(\varphi(e_j)) \\
    &= f^{*i}\left(\sum_{k=1}^{m} A_{kj} f_k\right) \\
    &= \sum_{k=1}^{m} A_{kj} f^{*i}(f_k) \\
    &= A_{ij}
\end{align}
一行目は$^t \varphi=g\circ \varphi$という表式を用いた。3行目は$f^{*i}(f_k)=\delta_k^i$を用いた。
\\したがって(\ref{4})式と見比べることにより、$B_{ji} = A_{ij}$ であり、行
列を入れ替えたもの、すなわち $B = {}^tA$ であることが示された。

\section{問題2}
\noindent
問2. $f: \mathbf{R}^2 \setminus \{(0,0)\} \longrightarrow \mathbf{R}^2 \setminus \{(0,0)\}$ を次式で定義される $C^\infty$ 級写像とする.
\[
  f(x, y) = (x^2 - y^2, 2xy).
\]
(1) $f_* \left( \frac{\partial}{\partial x} \right), f_* \left( \frac{\partial}{\partial y} \right)$ を計算せよ.

\vspace{1em}

(2) $\mathbf{R}^2 \setminus \{(0,0)\}$ 上の 1 次微分形式
\[
  \alpha = -\frac{y}{x^2 + y^2}dx + \frac{x}{x^2 + y^2}dy
\]
の $f$ による引き戻し $f^*\alpha$ を求めよ.
\subsection{解答}
(1) $f_*$ の定義に従って計算する。
\begin{align}
    f_*\left(\pdv{x}\right) &= \pdv{f^1}{x}\pdv{x^1} + \pdv{f^2}{x}\pdv{x^2} \\
    &= 2x\pdv{x^1} + 2y\pdv{x^2}\\
    f_*\left(\pdv{y}\right) &= \pdv{f^1}{y}\pdv{x^1} + \pdv{f^2}{y}\pdv{x^22} \\
    &= -2y\pdv{x^1} + 2x\pdv{x^2}
\end{align}
(2) $f^*$ の定義に従って計算する。
\begin{align}
    f^*\alpha &= -\frac{f^2(x,y)}{(f^1(x,y))^2 + (f^2(x,y))^2} d(f^1(x,y)) + \frac{f^1(x,y)}{(f^1(x,y))^2 + (f^2(x,y))^2} d(f^2(x,y)) \\
    &= -\frac{2xy}{(x^2 - y^2)^2 + (2xy)^2} d(x^2 - y^2) + \frac{x^2 - y^2}{(x^2 - y^2)^2 + (2xy)^2} d(2xy) \\
    &= -\frac{2xy}{(x^2 + y^2)^2} (2x dx - 2y dy) + \frac{x^2 - y^2}{(x^2 + y^2)^2} (2y dx + 2x dy) \\
    &= \left(-\frac{4x^2y}{(x^2 + y^2)^2} + \frac{2y(x^2 - y^2)}{(x^2 + y^2)^2}\right) dx + \left(\frac{4xy^2}{(x^2 + y^2)^2} + \frac{2x(x^2 - y^2)}{(x^2 + y^2)^2}\right) dy \\
    &= \frac{-2y(x^2+y^2)}{(x^2+y^2)^2} \cdot dx + \frac{2x(x^2+y^2)}{(x^2+y^2)^2}\cdot dy\\
    &= -\frac{2y}{(x^2+y^2)} dx + \frac{2x}{(x^2+y^2)} dy
\end{align}
よって示される。\\
また、他のやり方でも示してみる。
\begin{align}
  f^*\alpha \qty(\pdv{x})&=\alpha\qty(f_*\qty(\pdv{x}))\\
  &= \alpha\qty(2x\pdv{x^1} + 2y\pdv{x^2})\\
  &= -\frac{x^2}{(x^1)^2+(x^2)^2}2x + \frac{x^1}{(x^1)^2+(x^2)^2}2y\\
  &= -\frac{2xy}{(x^2 - y^2)^2 + (2xy)^2} (2x) + \frac{x^2 - y^2}{(x^2 - y^2)^2 + (2xy)^2} (2y)\\
  &= -\frac{2y(x^2+y^2)}{(x^2+y^2)^2}\\
  &= -\frac{2y}{(x^2+y^2)}
\end{align}
 で$dx$の係数が示される。
同様に、

\begin{align}
    f^*\alpha \qty(\pdv{y})&=\alpha\qty(f_*\qty(\pdv{y}))\\
    &= \alpha\qty(-2y\pdv{x^1} + 2x\pdv{x^2})\\
    &= -\frac{x^2}{(x^1)^2+(x^2)^2}(-2y) + \frac{x^1}{(x^1)^2+(x^2)^2}(2x)\\
    &=-\frac{2xy}{(x^2 - y^2)^2 + (2xy)^2} (-2y) + \frac{x^2 - y^2}{(x^2 - y^2)^2 + (2xy)^2} (2x)\\
    &=\frac{2x(x^2+y^2)}{(x^2+y^2)^2}\\
    &=\frac{2x}{(x^2+y^2)}
\end{align}
で$dy$の係数が示される。
つまり、$f^*\alpha = -\frac{2y}{(x^2+y^2)} dx + \frac{2x}{(x^2+y^2)} dy$ である。
\section{問題3}問3. $\mathbf{R}^4$ の座標を $(x, y, z, t)$ とし,
\begin{align*}
E_x &= E_x(x, y, z, t), & E_y &= E_y(x, y, z, t), & E_z &= E_z(x, y, z, t) \\
B_x &= B_x(x, y, z, t), & B_y &= B_y(x, y, z, t), & B_z &= B_z(x, y, z, t)
\end{align*}
を $\mathbf{R}^4$ 上の $C^{\infty}$ 級関数として,
$$
F = E_x dx \wedge dt + E_y dy \wedge dt + E_z dz \wedge dt + B_x dy \wedge dz + B_y dz \wedge dx + B_z dx \wedge dy \in \Omega^2(\mathbf{R}^4)
$$
とする.

(1) $dF$ を計算せよ. また $dF=0$ からどのような式が出てくるか?

(2) 写像 $* : \Omega^2(\mathbf{R}^4) \to \Omega^2(\mathbf{R}^4)$ を次のように定義する.
\begin{gather*}
*(dx \wedge dt) = dy \wedge dz, \quad *(dy \wedge dt) = dz \wedge dx, \quad *(dz \wedge dt) = dx \wedge dy, \\
*(dx \wedge dy) = -dz \wedge dt, \quad *(dy \wedge dz) = -dx \wedge dt, \quad *(dz \wedge dx) = -dy \wedge dt
\end{gather*}
とし, これを $C^{\infty}(\mathbf{R}^4)$-linear に $\Omega^2(\mathbf{R}^4) \to \Omega^2(\mathbf{R}^4)$ に拡張する.

例えば
$$
(f(x, y, z, t) dx \wedge dy + g(x, y, z, t) dz \wedge dt) = f(x, y, z, t) \{*(dx \wedge dy)\} + g(x, y, z, t) \{*(dz \wedge dt)\}
$$

といった具合である。

このとき, $d(*F)$ を計算せよ. また, $d(*F)=0$ からどのような式が出てくるか?
\subsection{解答}
(1) $dF$ を計算する。
\begin{align}
    dF &= d(E_x dx \wedge dt) + d(E_y dy \wedge dt) + d(E_z dz \wedge dt) \notag\\
    & \ \ + d(B_x dy \wedge dz) + d(B_y dz \wedge dx) + d(B_z dx \wedge dy) \\
    &= \left(\pdv{E_x}{y} dy + \pdv{E_x}{z} dz \right) \wedge dx \wedge dt 
     + \left(\pdv{E_y}{x} dx + \pdv{E_y}{z} dz \right) \wedge dy \wedge dt \notag\\
    & \ \ + \left(\pdv{E_z}{x} dx + \pdv{E_z}{y} dy\right) \wedge dz \wedge dt 
    + \left(\pdv{B_x}{x} dx + \pdv{B_x}{t} dt\right) \wedge dy \wedge dz \notag\\
    & \ \ + \left(\pdv{B_y}{y} dy  + \pdv{B_y}{t} dt \right) \wedge dz \wedge dx 
     + \left( \pdv{B_z}{z} dz + \pdv{B_z}{t} dt\right) \wedge dx \wedge dy\\
    &=\left(\pdv{B_z}{t}-\pdv{E_x}{y}+\pdv{E_y}{x}\right) dt\wedge dx\wedge dy+\left(\pdv{B_y}{t}+\pdv{E_x}{z}-\pdv{E_z}{x}\right) dt\wedge dz \wedge dx \notag\\
    &\ \ +\left(\pdv{B_x}{t}-\pdv{E_y}{z}+\pdv{E_z}{y}\right)dt\wedge dy\wedge dz+\qty(\pdv{B_x}{x}+\pdv{B_y}{y}+\pdv{B_z}{z})dx\wedge dy\wedge dz
\end{align}
dF=0 から次の式が出てくる。
\begin{align}
    \pdv{B_z}{t}-\pdv{E_x}{y}+\pdv{E_y}{x} &= 0 \label{34}\\
    \pdv{B_y}{t}+\pdv{E_x}{z}-\pdv{E_z}{x} &= 0 \label{35}\\
    \pdv{B_x}{t}-\pdv{E_y}{z}+\pdv{E_z}{y} &= 0 \label{36}\\
    \pdv{B_x}{x}+\pdv{B_y}{y}+\pdv{B_z}{z} &= 0\label{37}
\end{align}
(\ref{34})(\ref{35})(\ref{36})(\ref{37})式より次のマクスウェル方程式が得られる。
\begin{align}
    \nabla \times \mathbf{E} &= -\pdv{\mathbf{B}}{t} \\
    \nabla \cdot \mathbf{B} &= 0
\end{align}
\\
\\
(2) $*F$ を計算する。
定義に従って計算する。
\begin{align}
    *F &= *(E_x dx \wedge dt) + *(E_y dy \wedge dt) + *(E_z dz \wedge dt) \notag\\
    & \ \ + *(B_x dy \wedge dz) + *(B_y dz \wedge dx) + *(B_z dx \wedge dy) \\
    &= E_x (dy \wedge dz) + E_y (dz \wedge dx) + E_z (dx \wedge dy) \notag\\
    & \ \ - B_x (dx \wedge dt) - B_y (dy \wedge dt) - B_z (dz \wedge dt)
\end{align}
これの外微分を計算する。
\begin{align}
    d(*F)=& d(E_x dy \wedge dz) + d(E_y dz \wedge dx) + d(E_z dx \wedge dy) \notag\\
    & - d(B_x dx \wedge dt) - d(B_y dy \wedge dt) - d(B_z dz \wedge dt) \\
    =& \left(\pdv{E_x}{x} dx + \pdv{E_x}{t} dt\right) \wedge dy \wedge dz 
     + \left(\pdv{E_y}{y} dy + \pdv{E_y}{t} dt\right) \wedge dz \wedge dx \notag\\
    & + \left(\pdv{E_z}{z} dz + \pdv{E_z}{t} dt\right) \wedge dx \wedge dy 
    - \left(\pdv{B_x}{y} dy + \pdv{B_x}{z} dz\right) \wedge dx \wedge dt \notag\\
    & - \left(\pdv{B_y}{z} dz + \pdv{B_y}{x} dx\right) \wedge dy \wedge dt 
     - \left( \pdv{B_z}{x} dx + \pdv{B_z}{y} dy\right) \wedge dz \wedge dt
\end{align}
それぞれの項を整理すると、
\begin{align}
    d(*F)=&\qty(\pdv{E_x}{x}+\pdv{E_y}{y}+\pdv{E_z}{z}) dx \wedge dy \wedge dz \notag\\
    & +\left(\pdv{E_z}{t}+\pdv{B_x}{y}-\pdv{B_y}{x}\right) dt \wedge dx \wedge dy \notag\\
    & +\left(\pdv{E_x}{t}+\pdv{B_y}{z}-\pdv{B_z}{y}\right) dt \wedge dy \wedge dz \notag\\
    & +\left(\pdv{E_y}{t}+\pdv{B_z}{x}-\pdv{B_x}{z}\right) dt \wedge dz \wedge dx 
\end{align}
d(*F)=0 から次の式が出てくる。
\begin{align}
    \pdv{E_x}{x}+\pdv{E_y}{y}+\pdv{E_z}{z} &= 0 \label{52}\\
    \pdv{E_z}{t}+\pdv{B_x}{y}-\pdv{B_y}{x} &= 0 \label{53}\\
    \pdv{E_x}{t}+\pdv{B_y}{z}-\pdv{B_z}{y} &= 0 \label{54}\\
    \pdv{E_y}{t}+\pdv{B_z}{x}-\pdv{B_x}{z} &= 0 \label{55}
\end{align}
(\ref{52})(\ref{53})(\ref{54})(\ref{55})式より次のソース項のない場合のマクスウェル方程式が得られる。
\begin{align}
    \nabla \times \mathbf{B} &= \pdv{\mathbf{E}}{t} \\
    \nabla \cdot \mathbf{E} &= 0
\end{align}

問 11. Stokes の定理を使って
\[
D = \{(x, y, z) \in \mathbf{R}^3 \mid x^2 + y^2 + z^2 < 1\}
\]
の体積
\[
\iiint_D dx dy dz
\]
を求めよ。
\subsection{解答}
Stokes の定理を使うために、 まず適切な 2 次微分形式を考える。 ここで。 次の 2 次微分形式を考える。
\begin{align}
    \omega = x dy \wedge dz + y dz \wedge dx + z dx \wedge dy
\end{align}
という微分形式を考える。
このとき、$d\omega$ を計算する。
\begin{align}   
    d\omega &= d(x dy \wedge dz) + d(y dz \wedge dx) + d(z dx \wedge dy) \\
    &= dx \wedge dy \wedge dz + dy \wedge dz \wedge dx + dz \wedge dx \wedge dy \\
    &= 3 dx \wedge dy \wedge dz
\end{align}
したがって、Stokes の定理より、
\begin{align}
    \iiint_D dx dy dz &= \frac{1}{3} \iiint_D d\omega \\
    &= \frac{1}{3} \iint_{\partial D} \omega
\end{align}
ここで、$\partial D$ は $D$ の境界であり、単位球面 $x^2 + y^2 + z^2 = 1$ である。 単位球面上での $\omega$ を計算する。
極座標変換を用いる。
\begin{align}
    \omega &= x dy \wedge dz + y dz \wedge dx + z dx \wedge dy \\
    &= \sin\theta \cos \phi d(\sin\theta \sin \phi) \wedge d(\cos \theta) 
    + \sin\theta \sin \phi d(\cos \theta) \wedge d(\sin\theta \cos \phi) \notag\\
    & \ \ + \cos \theta d(\sin\theta \cos \phi) \wedge d(\sin\theta \sin \phi)\\
    &= \sin\theta \cos \phi \sin\theta \cos\phi \cdot - \sin \theta d\phi \wedge d\theta  
     + \sin\theta \sin \phi \cdot-\sin \theta \cdot \sin\theta \cdot - \sin \phi d\theta \wedge d\phi \notag \\
    & \ \ + \cos \theta \cos \theta \cos \phi \sin \theta\cdot\cos \phi d\theta \wedge d\phi
        + \cos \theta \sin \theta \cdot -\sin \phi \cos \theta \sin \phi d\phi \wedge d\theta\\
        &= \sin^3 \theta d\theta \wedge d \phi+\sin \theta \cos^2\theta\cos^2\phi d\theta\wedge d\phi +\sin \theta \sin^2\theta\cos^2\phi d\theta \wedge d\phi\\
        &=\sin\theta d \theta \wedge d\phi
\end{align}
したがって、単位球面上での $\omega$ は $\sin\theta d \theta \wedge d\phi$ である。 これを用いて、次のように計算できる。
\begin{align}
    \iint_{\partial D} \omega &= \int_0^{2\pi} \int_0^{\pi} \sin\theta d\theta d\phi \\
    &= \int_0^{2\pi} d\phi \int_0^{\pi} \sin\theta d\theta \\
    &= 2\pi \cdot 2 = 4\pi
\end{align}
よって、求める体積は次のようになる。
\begin{align}
    \iiint_D dx dy dz &= \frac{1}{3} \iint_{\partial D} \omega \\
    &= \frac{1}{3} \cdot 4\pi = \frac{4\pi}{3}
\end{align}


\end{document}